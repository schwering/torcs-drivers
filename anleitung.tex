\documentclass[a4paper,10pt]{scrartcl}

\usepackage[ngerman]{babel}

\begin{document}

\begin{itemize}
\item Man darf vorher gerne "uben.
\item Das rote Auto erstmal drei bis f"unf Sekunden wegfahren lassen.
\item Beim anfahren darf man zuerst langsam beschleunigen (bis so
    40 km/h), dann soll man aber Vollgas geben und {\em nicht mehr vom
    Gas gehen}.
    Das Auto ist bei 75 km/h abgeriegelt.
\item Man darf ruhig eine Weile {\em auf der "Uberholspur bleiben};
    man sollte nicht erst in der letzten Sekunde ausscheren
    und sofort wieder einscheren.
\item Um auf einer Spur zu bleiben, sollte man {\em einfach geradeaus
    fahren}. Man muss nicht krampfhaft versuchen, in der Mitte
    der Spur zu bleiben. Geradeaus fahren reicht aus,
    wie im normalen Auto.
\item Bitte nach abgeschlossenem "Uberholman"over bei gleicher
    Geschwindigkeit mindestens {\em bis zur Kurve geradeaus} fahren,
    damit die Planerkennung noch etwas laufen kann.
    Danach darf man dann ausrollen lassen.
\item Beim rechts \"uberholen soll das System in die Falle gelockt
    werden. Deswegen m\"oglichst nah rechts am roten Auto
    vorbeifahren und dabei m"oglichst wenig schlingern.
\item In die rechte Spalte bitte die {\em h"ochste Wahrscheinlichkeit}
    eintragen, die w\"ahrend des Fahrens aufgetaucht ist.
    Das ist die Wahrscheinlichkeit, die etwas "uber der Mitte des
    Bildschirms in gro"ser, gr"uner oder roter Schrift erscheint.
\end{itemize}

\begin{center}
\begin{tabular}{|r|l|r|}
\hline
\bf \# & \bf Man"over & \bf Wahrscheinlichkeit\\ \hline
1 & links "uberholen & \\ \hline
2 & links "uberholen & \\ \hline
3 & links "uberholen & \\ \hline
4 & links "uberholen & \\ \hline
5 & rechts "uberholen & \\ \hline
6 & links "uberholen & \\ \hline
7 & links "uberholen & \\ \hline
8 & links "uberholen & \\ \hline
9 & links "uberholen & \\ \hline
10 & rechts "uberholen & \\ \hline
11 & links "uberholen & \\ \hline
12 & links "uberholen & \\ \hline
13 & links "uberholen & \\ \hline
14 & links "uberholen & \\ \hline
15 & rechts "uberholen & \\ \hline
16 & links "uberholen & \\ \hline
17 & links "uberholen & \\ \hline
18 & links "uberholen & \\ \hline
19 & links "uberholen & \\ \hline
20 & rechts "uberholen & \\ \hline
\end{tabular}
\end{center}

\end{document}

