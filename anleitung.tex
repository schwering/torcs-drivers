\documentclass[a4paper,12pt]{scrartcl}

\usepackage[ngerman]{babel}
\usepackage[latin1]{inputenc}
\usepackage[
  pdftex,a4paper=true,colorlinks=true,
  pdftitle={Formular},pdfsubject={Zugang},
  pdfauthor={ich},pdfpagemode=UseNone,pdfstartview=FitH,
  pagebackref,pdfhighlight={/N}
]{hyperref}



\begin{document}

\section*{Tipps}
\begin{itemize}
\item Man darf vorher gerne "uben.
\item Leider hat das Auto leichten Drall. Das ist nicht gewollt.
    Es ist ein Fehler im Spiel und au"serdem ist das Lenkrad schwer
    zu kalibrieren.
\item Beim Start wird in gro"ser Schrift eingeblendet, wann man losfahren
    soll.
\item Beim anfahren darf man zun"achst langsam beschleunigen (bis etwa
    40 km/h), dann soll man aber Vollgas geben und {\em nicht mehr vom
    Gas gehen}.
    Das Auto ist bei 75 km/h abgeriegelt.
\item Man darf ruhig eine Weile {\em auf der "Uberholspur bleiben};
    man sollte nicht erst in der letzten Sekunde ausscheren
    und sofort wieder einscheren.
\item Um auf einer Spur zu bleiben, sollte man {\em einfach geradeaus
    fahren}. Man muss nicht krampfhaft versuchen, in der Mitte
    der Spur zu bleiben. Geradeaus fahren reicht aus,
    wie im normalen Auto.
\item Bitte nach abgeschlossenem "Uberholman"over bei gleicher
    Geschwindigkeit mindestens {\em bis zur Kurve geradeaus} fahren,
    damit die Planerkennung noch etwas laufen kann.
    Danach darf man dann ausrollen lassen.
\item Beim rechts \"uberholen soll das System in die Falle gelockt
    werden. Deswegen m\"oglichst nah rechts am roten Auto
    vorbeifahren und dabei m"oglichst wenig schlingern.
\item In die rechte Spalte bitte die {\em h"ochste Wahrscheinlichkeit}
    eintragen, die w\"ahrend des Fahrens aufgetaucht ist.
    Das ist die Wahrscheinlichkeit, die etwas "uber der Mitte des
    Bildschirms in gro"ser, gr"uner oder roter Schrift erscheint.
\end{itemize}

\section*{Tabelle}
\hspace{\parindent}
\begin{Form}[action=mailto:schwering@gmail.com?subject=TORCS-Evaluation,encoding=html,method=post]
%\begin{figure}[ht]
%
\begin{minipage}{0.5\linewidth}
\begin{tabular}{|r|l|r|}
\hline
\bf \# & \bf Man"over  & \bf Konfidenz\\ \hline
1 & links "uberholen   & \TextField[name=m1,width=0.2\linewidth]{} \%\\ \hline
2 & links "uberholen   & \TextField[name=m2,width=0.2\linewidth]{} \%\\ \hline
3 & links "uberholen   & \TextField[name=m3,width=0.2\linewidth]{} \%\\ \hline
4 & links "uberholen   & \TextField[name=m4,width=0.2\linewidth]{} \%\\ \hline
5 & rechts "uberholen  & \TextField[name=m5,width=0.2\linewidth]{} \%\\ \hline
6 & links "uberholen   & \TextField[name=m6,width=0.2\linewidth]{} \%\\ \hline
7 & links "uberholen   & \TextField[name=m7,width=0.2\linewidth]{} \%\\ \hline
8 & links "uberholen   & \TextField[name=m8,width=0.2\linewidth]{} \%\\ \hline
9 & links "uberholen   & \TextField[name=m9,width=0.2\linewidth]{} \%\\ \hline
10 & rechts "uberholen & \TextField[name=m10,width=0.2\linewidth]{} \%\\ \hline
\end{tabular}
\end{minipage}
%
\begin{minipage}{0.5\linewidth}
\begin{tabular}{|r|l|r|}
\hline
\bf \# & \bf Man"over  & \bf Konfidenz\\ \hline
11 & links "uberholen  & \TextField[name=m11,width=0.2\linewidth]{} \%\\ \hline
12 & links "uberholen  & \TextField[name=m12,width=0.2\linewidth]{} \%\\ \hline
13 & links "uberholen  & \TextField[name=m13,width=0.2\linewidth]{} \%\\ \hline
14 & links "uberholen  & \TextField[name=m14,width=0.2\linewidth]{} \%\\ \hline
15 & rechts "uberholen & \TextField[name=m15,width=0.2\linewidth]{} \%\\ \hline
16 & links "uberholen  & \TextField[name=m16,width=0.2\linewidth]{} \%\\ \hline
17 & links "uberholen  & \TextField[name=m17,width=0.2\linewidth]{} \%\\ \hline
18 & links "uberholen  & \TextField[name=m18,width=0.2\linewidth]{} \%\\ \hline
19 & links "uberholen  & \TextField[name=m19,width=0.2\linewidth]{} \%\\ \hline
20 & rechts "uberholen & \TextField[name=m20,width=0.2\linewidth]{} \%\\ \hline
\end{tabular}
\end{minipage}
%
\begin{minipage}[b]{\linewidth}
\centering
\vspace{3mm}
\Submit{Abschicken}
\end{minipage}
%
%\end{figure}
\end{Form}

\end{document}

