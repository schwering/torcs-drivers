\documentclass[a4paper,10pt]{scrartcl}

\usepackage[ngerman]{babel}

\begin{document}

\begin{itemize}
\item Das rote Auto erstmal drei bis f"unf Sekunden wegfahren lassen.
\item Beim anfahren kann man anfangs langsam beschleunigen, dann bitte
        Vollgas geben. Ab dann nicht mehr vom Gas gehen.
\item Man darf ruhig ein Weilchen auf der linken Spur bleiben; man
        sollte nicht erst in der letzten Sekunde ausscheren und sofort
        wieder einscheren.
\item Um auf einer Spur zu bleiben, sollte man einfach geradeaus
        fahren. Man muss nicht krampfhaft versuchen, in der Mitte
        der Spur zu bleiben. Einfach geradeaus fahren reicht aus,
        wie im normalen Auto.
\item Bitte nach abgeschlossenem "Uberholman"over bei gleicher
        Geschwindigkeit weiter {\em geradeaus} fahren, damit die
        Planerkennung noch etwas laufen kann.
\item Beim rechts \"uberholen soll das System in die Falle gelockt
        werden. Deswegen m\"oglichst nah rechts am roten Auto
        vorbeifahren und dabei m"oglichst wenig schlingern.
\item In die rechte Spalte bitte die {\em h"ochste} Wahrscheinlichkeit
        eintragen, die w\"ahrend des Fahrens aufgetaucht ist.
\end{itemize}

\begin{center}
\begin{tabular}{|r|l|r|}
\hline
\bf \# & \bf Man"over & \bf Wahrscheinlichkeit\\ \hline
1 & links "uberholen & \\ \hline
2 & links "uberholen & \\ \hline
3 & links "uberholen & \\ \hline
4 & links "uberholen & \\ \hline
5 & rechts "uberholen & \\ \hline
6 & links "uberholen & \\ \hline
7 & links "uberholen & \\ \hline
8 & links "uberholen & \\ \hline
9 & links "uberholen & \\ \hline
10 & rechts "uberholen & \\ \hline
11 & links "uberholen & \\ \hline
12 & links "uberholen & \\ \hline
13 & links "uberholen & \\ \hline
14 & links "uberholen & \\ \hline
15 & rechts "uberholen & \\ \hline
16 & links "uberholen & \\ \hline
17 & links "uberholen & \\ \hline
18 & links "uberholen & \\ \hline
19 & links "uberholen & \\ \hline
20 & rechts "uberholen & \\ \hline
\end{tabular}
\end{center}

\end{document}

